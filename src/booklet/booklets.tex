% options:
%\def\LETTERPAPER{} % use letter page
%\def\AFOURPAPER{} % use A4 paper
%otherwise standalone environment

\ifdefined\LETTERPAPER
\documentclass{article}
\usepackage[letterpaper,margin=1in]{geometry}
\pagestyle{empty}
\def\FULLPAGE{}
\else
    \ifdefined\AFOURPAPER
	\documentclass{article}
    \usepackage[a4paper,margin=1in]{geometry}
	\pagestyle{empty}
	\def\FULLPAGE{}
	\else
	\documentclass{standalone}
	\fi
\fi

% options:
%\def\BLEEDAREA{} % put bleed area around image as required for printing at makeplayingcards.com https://www.makeplayingcards.com/dl/booklet-template/us-game-4pp.pdf
%\def\LINES{} % put lines for cutting
%\def\SAFEAREA{} % draw safe area for content

\newif\ifwhitecard
\whitecardtrue

% one of the following
%\def\BLACKBACKGROUNDGRAY{} % black cards have light gray background
%\def\BLACKBACKGROUNDWHITE{} % black cards have white background

%Packages
\usepackage{microtype}
\usepackage{tikz}
\usepackage{xcolor}
\usepackage{xspace}
\usepackage{verbatimbox}
% fonts
\usepackage{ulem}
\renewcommand{\ULthickness}{1.5pt}
\usepackage{wasysym}
\usepackage{fontspec}
\usepackage{amsfonts,amssymb}
\usepackage{dsfont}
\newfontfamily\comicsans{Comic Sans MS}
\setsansfont{Helvetica}

%version
\newcommand{\version}{1.0}

% font settings for default text
%Cards
\newcommand{\bodyfont}{\sffamily\bfseries\fontsize{12}{18}\selectfont}
\newcommand{\titlefont}{\sffamily\bfseries\fontsize{20.5}{22}\selectfont}
%Booklet
\newcommand{\booklettitlefont}{\sffamily\bfseries\fontsize{20.5}{25}\selectfont}
\newcommand{\bookletfont}{\sffamily\fontsize{12}{12}\selectfont}
\newcommand{\bookletfontsmall}{\sffamily\fontsize{8}{8}\selectfont}
\newcommand{\bookletfonttiny}{\sffamily\fontsize{6}{6}\selectfont}

% prevent hyphenation
\pretolerance=10000
\tolerance=2000 
\emergencystretch=10pt

% color scheme
\newcommand{\setcolorscheme}{% This macro allows to switch white/black within a document
\ifwhitecard %Answer card
	\definecolor{cardbg}{named}{white}
	\definecolor{cardfg}{named}{black}
	\definecolor{icondark}{named}{black}
	\definecolor{iconlight}{named}{white}
\else %Sentence card
	\ifdefined\BLACKBACKGROUNDWHITE %White background
	\definecolor{cardbg}{named}{white}
	\definecolor{cardfg}{named}{black}
	\definecolor{icondark}{named}{black}
	\definecolor{iconlight}{named}{white}
	\else
		\ifdefined\BLACKBACKGROUNDGRAY %Gray background
		\definecolor{cardbg}{rgb}{0.85,0.85,0.85}
		\definecolor{cardfg}{named}{black}
		\definecolor{icondark}{named}{black}
		\definecolor{iconlight}{named}{white}
		\else %Black background
		\definecolor{cardbg}{named}{black}%{cmyk}{0.4,0,0,1}
		\definecolor{cardfg}{named}{white}
		\definecolor{icondark}{named}{black}
		\definecolor{iconlight}{named}{white}
		\fi
	\fi	
\fi
}
\setcolorscheme

% Layout Elements
%BLANK
\newcommand{\BLANK}[1][6ex]{\raisebox{-0.15ex}{{\tikz\draw[line width=0.175ex,color=cardfg] (0,0)--+(#1,0);}}\xspace}
%Censor
\newcommand{\CENSOR}[1][8ex]{\raisebox{-0.41ex}{{\tikz\draw[line width=1.81ex,color=cardfg] (0,0)--+(#1,0);}}\xspace}
%Circles
\newcommand{\circled}[1]{\tikz[baseline=(c.base)]{%
\node[cardfg,draw=cardfg,shape=circle,inner sep=0.2ex,line width=0.3ex,text width=3ex,align=center] (c) {\bfseries\sffamily#1};
}}
\newcommand{\circlef}[1]{\tikz[baseline=(c.base)]{%
\node[cardbg,fill=cardfg,shape=circle,inner sep=0ex,text width=3ex,align=center] (c) {\bfseries\sffamily#1};
}}
%Draw N Command
\newcommand{\drawN}[1]{PICK \circlef{\large#1}}
%from https://groups.google.com/forum/?fromgroups=#!topic/comp.text.tex/aaDphoWpbwQ
\def\bitcoinA{%
  \leavevmode
  \vtop{\offinterlineskip %\bfseries
    \setbox0=\hbox{B}%
    \setbox2=\hbox to\wd0{\hfil\hskip-.03em
    \vrule height .3ex width .15ex\hskip .08em
    \vrule height .3ex width .15ex\hfil}
    \vbox{\copy2\box0}\box2}}
%The CAC icon
\newcommand{\cardicon}{
	\begin{tikzpicture}[yscale=-1]
	\draw [color=cardfg,fill=icondark,line width=0.5pt,rotate around={-20:(0.19in,0.105in)}] (0.075in,0.105in) rectangle +(0.19in,0.19in);
	\draw [color=icondark,fill=iconlight,line width=0.5pt] (0.19in,0.105in) rectangle +(0.19in,0.19in);
	\draw [color=icondark,fill=iconlight,line width=0.5pt,rotate around={20:(0.19in,0.105in)}] (0.275in,0.073in) rectangle +(0.19in,0.19in);
	\end{tikzpicture}
}
%Appendix logos
\newcommand{\appendixA}{\raisebox{.4ex}{Appendix} \huge $\mathds{A}$}
\newcommand{\appendixB}{\raisebox{.4ex}{Appendix} \huge $\bitcoinA$}


%CLASSIFIED REMOVE BEFORE PUBLICATION

\renewcommand{\baselinestretch}{0.8}
\renewcommand{\thefootnote}{\arabic{footnote}}

%Default measures for USCards booklets with bleedmargins for printing at makeplayingcards.com
%Check before printing!
\providecommand{\BOOKLETHALFWIDTH}{2.2in}
\providecommand{\BOOKLETWIDTH}{2*\BOOKLETHALFWIDTH}
\providecommand{\BOOKLETHEIGHT}{3.43in}
\providecommand{\BLEEDMARGIN}{0.12in}
\providecommand{\BOOKLETPADDING}{0.12in}%for safe area

\pgfdeclarelayer{bg}    % declare background layer
\pgfsetlayers{bg,main}  % set the order of the layers (main is the standard layer)

%Background
\newcommand{\drawbackground}{
\begin{pgfonlayer}{bg}
	\ifdefined\BLEEDAREA
	% background to bleed area
	\fill[cardbg] (-\BLEEDMARGIN, -\BLEEDMARGIN) rectangle (\BOOKLETWIDTH+\BLEEDMARGIN, \BOOKLETHEIGHT+\BLEEDMARGIN);
	\fi
	% background only to cut area
	\fill[cardbg] (0, 0) rectangle (\BOOKLETWIDTH, \BOOKLETHEIGHT);
\end{pgfonlayer}
	\ifdefined\LINES
	% guideline - cut area
	\draw[cardfg,line width=0.1mm] (0,0) rectangle (\BOOKLETWIDTH, \BOOKLETHEIGHT);
	\draw[cardfg,line width=0.1mm,dashed] (\BOOKLETHALFWIDTH, 0) -- (\BOOKLETHALFWIDTH, \BOOKLETHEIGHT);
	\fi
	\ifdefined\SAFEAREA
	% guideline - safe area
	\draw[red,dashed] (\BOOKLETPADDING, \BOOKLETPADDING) rectangle (\BOOKLETWIDTH - \BOOKLETPADDING, \BOOKLETHEIGHT - \BOOKLETPADDING);
	\draw[red,line width=0.1mm,dashed] (\BOOKLETHALFWIDTH-\BOOKLETPADDING, \BOOKLETPADDING) -- (\BOOKLETHALFWIDTH-\BOOKLETPADDING, \BOOKLETHEIGHT-\BOOKLETPADDING);
	\draw[red,line width=0.1mm,dashed] (\BOOKLETHALFWIDTH+\BOOKLETPADDING, \BOOKLETPADDING) -- (\BOOKLETHALFWIDTH+\BOOKLETPADDING, \BOOKLETHEIGHT-\BOOKLETPADDING);
	\fi
}

%Textbox
\newcommand{\drawbodytext}[4]{
\begin{scope}[shift={(#1,#2)}]
	\node [cardfg,below right,text width=#3] at (0,0) {#4};
\end{scope}
}

%horizontal line
\newcommand{\hrline}[1][\BOOKLETHALFWIDTH-3*\BOOKLETPADDING]{%
\tikz \draw [line width=0.2ex,color=cardfg] (0,0)--+(#1,0);
}

%title page elements
\newcommand{\gametitle}{%
	\node [cardfg,below right,text width=\BOOKLETHALFWIDTH - 2*\BOOKLETPADDING] at (\BOOKLETPADDING, \BOOKLETPADDING) {\booklettitlefont Cards \\ Against \\ Cryptography\par};
}
\newcommand{\drawgameicon}{
	\node (L) [cardfg,above right, text centered] at (\BOOKLETPADDING, \BOOKLETHEIGHT - \BOOKLETPADDING) {\cardicon};
}
\newcommand{\drawextralogo}[1]{
	\node (EL) [cardfg,above left, text centered] at (\BOOKLETHALFWIDTH - \BOOKLETPADDING, \BOOKLETHEIGHT - \BOOKLETPADDING) {\bodyfont #1 \par};
}
%title page
\newcommand{\frontpage}[3]{
\begin{scope}[shift={(#1,#2)}]
   \gametitle
   \drawgameicon
   \drawextralogo{#3}
\end{scope}
}

%Doublepage
\newenvironment{doublepage}{%
\noindent
\ifdefined\FULLPAGE
\begin{center}
\fi
\begin{tikzpicture}[yscale=-1]
\drawbackground
}{%
\end{tikzpicture}
\ifdefined\FULLPAGE
\end{center}
\newpage
\fi
}

\ifdefined\FULLPAGE{}
\else
	\standaloneenv{doublepage}
\fi


\begin{document}

%Outside Appendix A booklet
\begin{doublepage}
	% Title page (on the right)
	\frontpage{\BOOKLETHALFWIDTH}{0}{\appendixA}

	% back page
	\drawbodytext{\BOOKLETPADDING}{\BOOKLETPADDING}{\BOOKLETHALFWIDTH-2*\BOOKLETPADDING}{%
		{\bookletfontsmall
		Cards Against Cryptography - Appendix A is an independently made extension for Cards Against Cryptography 
		which itself is shamelessly based on Cards Against Humanity, which was released under a Creative Commons BY-NC-SA 2.0 license.}

		\medskip

		{\bookletfontsmall
		Cards Against Cryptography - Appendix A is released under the same license, which means you can use, remix, and share the game for free, but cannot sell it.}

		\bigskip

		\includegraphics[width=0.25in]{icons/cc}
		\includegraphics[width=0.25in]{icons/by}
		\includegraphics[width=0.25in]{icons/nc}
		\includegraphics[width=0.25in]{icons/sa}

		\bigskip

		{\bookletfontsmall
		The source code and printable PDF versions are available online at https://github.com/CACappendix/.  
		}

		\bigskip

		{\bookletfonttiny
		Version \circlef{\version} 
		}

	};
\end{doublepage}


\begin{doublepage}
	
	% first page
	\drawbodytext{\BOOKLETPADDING}{\BOOKLETPADDING}{\BOOKLETHALFWIDTH-2*\BOOKLETPADDING}{%
		
		{\bookletfont
			\textbf{Basic Rules}
		}
		
		\smallskip
		
		{\bookletfontsmall
	The basic rules are equivalent to the Cards Against Cryptography rules and are available online at\\[1.1ex] https://github.com/CardsAgainst\\Cryptography/.  
		}
	
		\smallskip	
		{\hrline}
		
		\medskip
		
		{\bookletfontsmall
			\drawN{${n}$} 
		}	


		\smallskip
		
		{\bookletfontsmall
		Some cards say {\bookletfonttiny{PICK}} {$n$} on the bottom. 
		}

		\smallskip
		
		{\bookletfontsmall
		To answer these, each player draws $n$ additional white cards from the white pile, 
		and plays a combination of $n+1$  cards. 
		Cards should be played in the order the Session Chair should read them --order matters. 
		}

		\smallskip
		
		{\bookletfontsmall
		 In the case of mutually distrustful players, card-based cryptography techniques can be used to combat malicious behaviors. 
		 }

{\hrline}
		
};
% second page
\drawbodytext{\BOOKLETHALFWIDTH+\BOOKLETPADDING}{\BOOKLETPADDING}{\BOOKLETHALFWIDTH-2*\BOOKLETPADDING}{%
%		\hrline
		{\bookletfont
			\textbf{University Regulations}
		}
		
		\medskip
			
		{\bookletfontsmall
		\textbf{Virtual Random Fellow}\textsuperscript{1}: 
		Every round, pick the appropriate number of white cards from the pile and place them into play. 
		These cards belong to the VRF --a simulated player-- and if they win the game, every player must admit that their submission did not improve the state of the art. 
		}	
		
		\medskip
		
		{\bookletfontsmall
			\textbf{Cryptographic Haikus}:  Don't need to follow the 5-7-5 form. They just have to be read dramatically.
		}
		
		\medskip
		
		{\bookletfontsmall
			\textbf{Reboot}: At any time, payers may trade in one won black card to return as many white cards as they'd like to the deck and draw back up to 7.
		}
	\bigskip
	\medskip
	\smallskip
	
	\hrule width0.3\textwidth
	\smallskip
		{	
		\bookletfonttiny
		\textsuperscript{1}\emph{This variant assumes semi-honest players.}
	}
	
	};
\end{doublepage}

%Outside Appendix B booklet
\begin{doublepage}
	% Title page (on the right)
	\frontpage{\BOOKLETHALFWIDTH}{0}{\appendixB}
	
	
	% back page
	\drawbodytext{\BOOKLETPADDING}{\BOOKLETPADDING}{\BOOKLETHALFWIDTH-2*\BOOKLETPADDING}{%
		{\bookletfontsmall
			Cards Against Cryptography - Appendix B is an independently made extension for Cards Against Cryptography 
			which itself is shamelessly based on Cards Against Humanity, which was released under a Creative Commons BY-NC-SA 2.0 license.}
		
		\medskip
		
		{\bookletfontsmall
			Cards Against Cryptography - Appendix B is released under the same license, which means you can use, remix, and share the game for free, but cannot sell it.}
		
		\bigskip
		
		\includegraphics[width=0.25in]{icons/cc}
		\includegraphics[width=0.25in]{icons/by}
		\includegraphics[width=0.25in]{icons/nc}
		\includegraphics[width=0.25in]{icons/sa}
		
		\bigskip
		
		{\bookletfontsmall
		The source code and printable PDF versions are available online at https://github.com/CACappendix/.  
		}
		
		\medskip
		
		{\bookletfonttiny
		Version \circlef{\version}
		}
		
	};
\end{doublepage}

\begin{doublepage}
	
	% first page
	\drawbodytext{\BOOKLETPADDING}{\BOOKLETPADDING}{\BOOKLETHALFWIDTH-2*\BOOKLETPADDING}{%
		
		{\bookletfont
			\textbf{Basic Rules}
		}
		
		\smallskip
		
		{\bookletfontsmall
	The basic rules are equivalent to the Cards Against Cryptography rules and are available online at\\[1.1ex] https://github.com/CardsAgainst\\Cryptography/.  
		}
		
		\hrline
		\medskip
		
				{\bookletfont
			\textbf{Blockchain cards}
		} 
	
	
	\medskip
		{\bookletfontsmall
		\textbf{Forks}: In case more than one blockchain card is played, players are requested to recognized their played cards.  
	}


\medskip 
			{\bookletfontsmall
		\textbf{DAG Chain}: In case more than one blockchain card is played, all players who played blockchain cards get a victory card. 
	}
	\hrline
\smallskip

		
	};
	% second page
	\drawbodytext{\BOOKLETHALFWIDTH+\BOOKLETPADDING}{\BOOKLETPADDING}{\BOOKLETHALFWIDTH-2*\BOOKLETPADDING}{%
		
		
		{\bookletfont
			\textbf{Alternative Mining Rules}
		} 
		
		
		\medskip
		{\bookletfontsmall
			\textbf{Gambling}: In case of ties, you can bet one of your black cards (from your victory pile) to play an additional white card.
			If you win, you get your point back; otherwise your point goes to the winner.	
		}
	
	\medskip
		
			{\bookletfontsmall
			\textbf{Byzantine Agreement}: In the absence of a trusted Session Chair the consensus protocol applies. 
			Each player picks their favorite card each round. The card with the most votes wins the round.
		}

		
	};
\end{doublepage}

\end{document}
