% options:
%\def\BLEEDAREA{} % put bleed area around image as required for printing at makeplayingcards.com https://www.makeplayingcards.com/dl/templates/us_game_deck.pdf
%\def\LINES{} % put lines for cutting
%\def\SAFEAREA{} % draw safe area for content

% one of the following
%\def\BLACKBACKGROUNDGRAY{} % black cards have light gray background
%\def\BLACKBACKGROUNDWHITE{} % black cards have white background

%Packages
\usepackage{microtype}
\usepackage{tikz}
\usepackage{xcolor}
\usepackage{xspace}
\usepackage{verbatimbox}
% fonts
\usepackage{ulem}
\renewcommand{\ULthickness}{1.5pt}
\usepackage{wasysym}
\usepackage{fontspec}
\usepackage{amsfonts,amssymb}
\usepackage{dsfont}
\newfontfamily\comicsans{Comic Sans MS}
\setsansfont{Helvetica}

%version
\newcommand{\version}{1.0}

% font settings for default text
%Cards
\newcommand{\bodyfont}{\sffamily\bfseries\fontsize{12}{18}\selectfont}
\newcommand{\titlefont}{\sffamily\bfseries\fontsize{20.5}{22}\selectfont}
%Booklet
\newcommand{\booklettitlefont}{\sffamily\bfseries\fontsize{20.5}{25}\selectfont}
\newcommand{\bookletfont}{\sffamily\fontsize{12}{12}\selectfont}
\newcommand{\bookletfontsmall}{\sffamily\fontsize{8}{8}\selectfont}
\newcommand{\bookletfonttiny}{\sffamily\fontsize{6}{6}\selectfont}

% prevent hyphenation
\pretolerance=10000
\tolerance=2000 
\emergencystretch=10pt

% color scheme
\newcommand{\setcolorscheme}{% This macro allows to switch white/black within a document
\ifwhitecard %Answer card
	\definecolor{cardbg}{named}{white}
	\definecolor{cardfg}{named}{black}
	\definecolor{icondark}{named}{black}
	\definecolor{iconlight}{named}{white}
\else %Sentence card
	\ifdefined\BLACKBACKGROUNDWHITE %White background
	\definecolor{cardbg}{named}{white}
	\definecolor{cardfg}{named}{black}
	\definecolor{icondark}{named}{black}
	\definecolor{iconlight}{named}{white}
	\else
		\ifdefined\BLACKBACKGROUNDGRAY %Gray background
		\definecolor{cardbg}{rgb}{0.85,0.85,0.85}
		\definecolor{cardfg}{named}{black}
		\definecolor{icondark}{named}{black}
		\definecolor{iconlight}{named}{white}
		\else %Black background
		\definecolor{cardbg}{named}{black}%{cmyk}{0.4,0,0,1}
		\definecolor{cardfg}{named}{white}
		\definecolor{icondark}{named}{black}
		\definecolor{iconlight}{named}{white}
		\fi
	\fi	
\fi
}
\setcolorscheme

% Layout Elements
%BLANK
\newcommand{\BLANK}[1][6ex]{\raisebox{-0.15ex}{{\tikz\draw[line width=0.175ex,color=cardfg] (0,0)--+(#1,0);}}\xspace}
%Censor
\newcommand{\CENSOR}[1][8ex]{\raisebox{-0.41ex}{{\tikz\draw[line width=1.81ex,color=cardfg] (0,0)--+(#1,0);}}\xspace}
%Circles
\newcommand{\circled}[1]{\tikz[baseline=(c.base)]{%
\node[cardfg,draw=cardfg,shape=circle,inner sep=0.2ex,line width=0.3ex,text width=3ex,align=center] (c) {\bfseries\sffamily#1};
}}
\newcommand{\circlef}[1]{\tikz[baseline=(c.base)]{%
\node[cardbg,fill=cardfg,shape=circle,inner sep=0ex,text width=3ex,align=center] (c) {\bfseries\sffamily#1};
}}
%Draw N Command
\newcommand{\drawN}[1]{PICK \circlef{\large#1}}
%from https://groups.google.com/forum/?fromgroups=#!topic/comp.text.tex/aaDphoWpbwQ
\def\bitcoinA{%
  \leavevmode
  \vtop{\offinterlineskip %\bfseries
    \setbox0=\hbox{B}%
    \setbox2=\hbox to\wd0{\hfil\hskip-.03em
    \vrule height .3ex width .15ex\hskip .08em
    \vrule height .3ex width .15ex\hfil}
    \vbox{\copy2\box0}\box2}}
%The CAC icon
\newcommand{\cardicon}{
	\begin{tikzpicture}[yscale=-1]
	\draw [color=cardfg,fill=icondark,line width=0.5pt,rotate around={-20:(0.19in,0.105in)}] (0.075in,0.105in) rectangle +(0.19in,0.19in);
	\draw [color=icondark,fill=iconlight,line width=0.5pt] (0.19in,0.105in) rectangle +(0.19in,0.19in);
	\draw [color=icondark,fill=iconlight,line width=0.5pt,rotate around={20:(0.19in,0.105in)}] (0.275in,0.073in) rectangle +(0.19in,0.19in);
	\end{tikzpicture}
}
%Appendix logos
\newcommand{\appendixA}{\raisebox{.4ex}{Appendix} \huge $\mathds{A}$}
\newcommand{\appendixB}{\raisebox{.4ex}{Appendix} \huge $\bitcoinA$}


%CLASSIFIED REMOVE BEFORE PUBLICATION

%Default measures for USCards with bleedmargins for printing at makeplayingcards.com
%Check before printing!
\providecommand{\CARDWIDTH}{2.2in}
\ifdefined\SQUARECARDS
\providecommand{\CARDHEIGHT}{\CARDWIDTH}
\else
\providecommand{\CARDHEIGHT}{3.43in}
\fi
\providecommand{\BLEEDMARGIN}{0.12in}
\providecommand{\CARDPADDING}{0.12in}%for safe area
\providecommand{\HEADERHEIGHT}{0.07in}
\providecommand{\TEXTMARGIN}{0.1in}
\providecommand{\FOOTERHEIGHT}{0.40in}
\pgfdeclarelayer{bg}    % declare background layer
\pgfsetlayers{bg,main}  % set the order of the layers (main is the standard layer)

%Background
\newcommand{\drawbackground}{
\begin{pgfonlayer}{bg}
	\ifdefined\BLEEDAREA
	% background to bleed area
	\fill[cardbg] (-\BLEEDMARGIN, -\BLEEDMARGIN) rectangle (\CARDWIDTH+\BLEEDMARGIN, \CARDHEIGHT+\BLEEDMARGIN);
	\fi
	% background only to cut area
	\fill[cardbg] (0, 0) rectangle +(\CARDWIDTH, \CARDHEIGHT);
\end{pgfonlayer}
	\ifdefined\LINES
	% guideline - cut area
	\draw[cardfg] (0, 0) rectangle +(\CARDWIDTH, \CARDHEIGHT);
	\fi
	\ifdefined\SAFEAREA
	% guideline - safe area
	\draw[red,dashed] (\CARDPADDING, \CARDPADDING) rectangle (\CARDWIDTH - \CARDPADDING, \CARDHEIGHT - \CARDPADDING);
	\fi
}
%Layout elements
\newcommand{\drawcardicon}{
	\node (L) [cardfg,above right] at (\CARDPADDING, \CARDHEIGHT - \CARDPADDING) {\cardicon};
}
\newcommand{\drawextralogo}[1]{
	\node (EL) [cardfg,above left] at (\CARDWIDTH - \CARDPADDING, \CARDHEIGHT - \CARDPADDING) {\bodyfont #1};
}
\newcommand{\drawbodytext}[1]{
	\node [cardfg,below right,text width=\CARDWIDTH - 2*\CARDPADDING-\TEXTMARGIN] at (\CARDPADDING, \CARDPADDING+\HEADERHEIGHT) {\bodyfont #1 \par};
}
\newcommand{\drawadvice}[1]{
	\node [cardfg,above left,align=right,text width=\CARDWIDTH - 2*\CARDPADDING-\TEXTMARGIN] at (\CARDWIDTH - \CARDPADDING, \CARDHEIGHT - \CARDPADDING - \FOOTERHEIGHT) {\bodyfont \small #1 \par};
}
\newcommand{\drawcardtitle}{
	\node [cardfg,below right,text width=\CARDWIDTH - 2*\CARDPADDING] at (\CARDPADDING, \CARDPADDING) {\titlefont Cards \\ Against \\ Cryptography\par};
}

%Cards
\newcommand{\cardfront}[5]{
\begin{scope}[shift={(#1,#2)}]
   \drawbackground
   \drawcardicon
   \drawbodytext{#3}
   \drawadvice{#4}
   \drawextralogo{#5}
\end{scope}
}
\newcommand{\cardback}[2]{
\begin{scope}[shift={(#1,#2)}]
   \drawbackground
   \drawcardtitle
\end{scope}
}